\section{Theorie}
\label{sec:Theorie}
In diesem Versuch wird die Dopplersonographie auf ihre Anwendung untersucht.
Unter Dopplersonographie wird die Untersuchung von Strömungen mittels Schall, im Besonderen Ultraschall, verstanden.
Dazu wird die Schallquelle an den Strom (meist ein Flüssigkeitsstrom) gekoppelt.
Wird der Schall nun an Teilchen des Stroms reflektiert, so verschiebt sich die Frequenz der Schallwelle wie durch den
Doppler-Effekt beschrieben.
Bewegt sich die Schallquelle der Frequenz $\nu_0$ mit Geschwindigkeit $v$ auf den Beobachter zu,
während dieser im Medium mit Schallgeschwindigkeit $c$ ruht, folgt die Frequenz, die der Beobachter misst nach:
\begin{equation}
	\nu = \frac{\nu_0}{1 - \frac{v}{c}}
\end{equation}
Für den Fall, dass die Quelle im Medium ruht und der Beobachter sich mit Geschwindigkeit $v$ zur Quelle hin bewegt gilt:
\begin{equation}
	\nu = \nu_0\left(1+\frac{v}{c}\right)
\end{equation}
Für die Dopplersonographie müssen nun beide Effekte kombiniert werden, da der Schall zunächst ein im Medium ruhendes Objekt erreicht,
und anschließend nach der Reflektion einen vom Medium aus bewegten Sensor trifft.
Berücksichtigt man zusätzlich den Winkel, den Quelle und Sensor zum Strom haben folgt:
\begin{equation}
	\label{eq:geschw}
	\symup\Delta\nu = 2\nu_0\frac{v}{c}\cos \alpha
\end{equation}
Die Gleichung\eqref{eq:geschw} kann in dieser Form aufgestellt werden, weil die Ultraschallsonde als Quelle und Sender fungiert.
Für die Erzeugung von Ultraschall wird der piezoelektrische Effekt genutzt.
Hierbei werden Kristalle mit einer speziellen Geometrie in ein elektrisches Wechselfeld gebracht,
und durch die Ladungsverschiebung ändern diese ihre Ausmaße geringfügig.
Durch Ausnutzung dieses Effektes ist es möglich hochfrequenten Schall zu erzeugen.
Wird der Kristall zusätzlich mit seiner Resonanzfrequenz an die gewünschte Frequenz angepasst,
ist dieses Verfahren zudem sehr effizient.
Gleichzeitig kann ein piezoelektrischer Kristall auch als Sensor genutzt werden, indem durch eintreffende Schallwellen
elektrische Signale erzeugt werden.
