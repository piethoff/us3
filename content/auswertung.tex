\section{Auswertung}
\label{sec:Auswertung}
\subsection{Bestimmung der Strömungsgeschwindigkeit}


Die Dopplerwinkel werden wie folgt berechnet:
\begin{equation}
  \alpha = 90\si{\degree} - \arcsin\left(\sin(\theta )\cdot \frac{c_L}{c_P}\right)
\end{equation}
Die erhaltenen Werte sind in Tabelle \ref{tab:dopp} eingetragen.
\begin{table}[H]
    \caption{Dopplerwinkel.}
    \label{tab:dopp}
    \centering
    \begin{tabular}{S[table-format=2] S[table-format=2.2(0)e0]  }
        \toprule
        {$\theta/\si{\degree}$} & {$\alpha/\si{\degree}$}  \\
        \midrule
             15 & 80.06 \\
             30 & 70.53\\
             60 & 54.74\\

        \bottomrule
    \end{tabular}
\end{table}
\noindent Die erhaltenen Werte der Messung sind in Tabellen \ref{tab:15}, \ref{tab:30} und \ref{tab:60} eingetragen.
Die Strömungsgeschwindigkeit wird nach Gleichung\eqref{eq:geschw} mit
\begin{equation}
 v = \frac{c  \symup\Delta\nu}{2\nu_0 \cos \alpha}
\end{equation}
berechnet.
%tabelle


\begin{table}
    \centering
    \caption{Messwerte für einen Winkel von $\alpha = \SI{15}{\degree}$.}
    \label{tab:15}
    \begin{tabular}{S[table-format=4.0] S[table-format=3.0] S[table-format=1.2] S[table-format=2.0]}
        \toprule
        {Strömungsgeschwindigkeit/RPM} & {$\symup\Delta\nu/\si{\hertz}$} & {$v/\si{\meter\per\second}$} & {Innendurchmesser$/\si{\milli\meter}$} \\
        \midrule
        3120  & 110 & 0.28 & 10\\
        3660  & 171 & 0.45 & 10\\
        4200  & 208 & 0.54 & 10\\
        4750  & 220 & 0.57 & 10\\
        5290  & 220 & 0.57 & 10\\
        3110  & 256 & 0.67 & 7\\
        3650  & 293 & 0.76 & 7\\
        4180  & 391 & 1.02 & 7\\
        4750  & 439 & 1.14 & 7\\
        5300  & 525 & 1.37 & 7\\
        3110  & 73 & 0.19 & 16\\
        3660  & 85 & 0.22 & 16\\
        4200  & 98 & 0.26 & 16\\
        4760  & 110 & 0.29 & 16\\
        5300  & 122 & 0.32 & 16\\

        \bottomrule
    \end{tabular}
\end{table}

\begin{table}
    \centering
    \caption{Messwerte für einen Winkel von $\alpha = \SI{30}{\degree}$.}
    \label{tab:30}
    \begin{tabular}{S[table-format=4.0] S[table-format=3.0] S[table-format=1.2] S[table-format=2.0]}
        \toprule
        {Strömungsgeschwindigkeit/RPM} & {$\symup\Delta\nu/\si{hertz}$}  & {$v/\si{\meter\per\second}$} & {Innendurchmesser$/\si{\milli\meter}$}\\
        \midrule
3120  & 220 & 0.30 & 10\\
3660  & 287 & 0.39 & 10\\
4200  & 305 & 0.41 & 10\\
4750  & 385 & 0.52 & 10\\
5290  & 385 & 0.52 & 10\\
3110  & 415 & 0.56 & 7\\
3650  & 458 & 0.62 & 7\\
4180  & 623 & 0.84 & 7\\
4750  & 580 & 0.78 & 7\\
5300  & 830 & 1.12 & 7\\
3110  & 92 & 0.12 & 16\\
3660  & 122 & 0.16 & 16\\
4200  & 134 & 0.18 & 16\\
4760  & 171 & 0.23 & 16\\
5300  & 195 & 0.26 & 16\\
        \bottomrule
    \end{tabular}
\end{table}

\begin{table}
    \centering
    \caption{Messwerte für einen Winkel von $\alpha = \SI{60}{\degree}$.}
    \label{tab:60}
    \begin{tabular}{S[table-format=4.0] S[table-format=3.0] S[table-format=1.2] S[table-format=2.0]}
        \toprule
        {Strömungsgeschwindigkeit/RPM} & {$\symup\Delta\nu/\si{hertz}$} & {$v/\si{\meter\per\second}$} & {Innendurchmesser$/\si{\milli\meter}$} \\
        \midrule
3120  & 305 & 0.23 & 10\\
3660  & 433 & 0.34 & 10\\
4200  & 531 & 0.41 & 10\\
4750  & 433 & 0.34 & 10\\
5290  & 647 & 0.50 & 10\\
3110  & 525 & 0.41 & 7\\
3650  & 702 & 0.55 & 7\\
4180  & 830 & 0.65 & 7\\
4750  & 1001 & 0.78 & 7\\
5300  & 1172 & 0.91 & 7\\
3110  & 134 & 0.10 & 16\\
3660  & 183 & 0.14 & 16\\
4200  & 220 & 0.17 & 16\\
4760  & 256 & 0.20 & 16\\
5300  & 281 & 0.22 & 16\\
        \bottomrule
    \end{tabular}
\end{table}

\begin{figure}[H]
  \centering
  \includegraphics{build/15.pdf}
  \caption{Geschwindigkeit aufgetragen gegen die Frequenzverschiebung geteilt durch die entsprechenden Dopplerwinkel für einen Rohrdurchmesser von $d= 10\si{\milli\meter}$.}
  \label{fig:15}
\end{figure}

\begin{figure}[H]
  \centering
  \includegraphics{build/30.pdf}
  \caption{Geschwindigkeit aufgetragen gegen die Frequenzverschiebung geteilt durch die entsprechenden Dopplerwinkel für einen Rohrdurchmesser von $d= 7\si{\milli\meter}$.}
  \label{fig:30}
\end{figure}

\begin{figure}[H]
  \centering
  \includegraphics{build/60.pdf}
  \caption{Geschwindigkeit aufgetragen gegen die Frequenzverschiebung geteilt durch die entsprechenden Dopplerwinkel für einen Rohrdurchmesser von $d= 16\si{\milli\meter}$.}
  \label{fig:60}
\end{figure}



\noindent In Abbildung \ref{fig:15}, \ref{fig:30} und \ref{fig:60} sind die Geschwindigkeiten gegen den Quotienten aus gemessener Frequenzänderung und dem Cosinus des jeweiligen
Dopplerwinkels aufgetragen.
%Zudem wird mit SciPy eine lineare Regression mit
%\begin{equation}
%  f(x)= mx + b
%\end{equation}
%durgeführt.
%Es ergeben sich folgende Fit-Parameter:
%\begin{align*}
%	\alpha = \SI{15}{\degree} 	& \\
%								& \text{Rohrdurchmesser}=\SI{7}{\milli\meter} 	& m=&\SI{0.72\pm0.05}{\hertz\per RPM}\\
%								&												& b=&\SI{-830\pm230}{\hertz}\\
%								& \text{Rohrdurchmesser}=\SI{10}{\milli\meter} 	& m=&\SI{0.29\pm0.08}{\hertz\per RPM}\\
%								&												& b=&\SI{-130\pm330}{\hertz}\\
%								& \text{Rohrdurchmesser}=\SI{16}{\milli\meter} 	& m=&\SI{0.130\pm0.001}{\hertz\per RPM}\\
%								&												& b=&\SI{19\pm5}{\hertz}\\
%	\alpha = \SI{30}{\degree} 	& \\
%								& \text{Rohrdurchmesser}=\SI{7}{\milli\meter} 	& m=&\SI{0.52\pm0.13}{\hertz\per RPM}\\
%								&												& b=&\SI{-400\pm500}{\hertz}\\
%								& \text{Rohrdurchmesser}=\SI{10}{\milli\meter} 	& m=&\SI{0.24\pm0.04}{\hertz\per RPM}\\
%								&												& b=&\SI{-50\pm160}{\hertz}\\
%								& \text{Rohrdurchmesser}=\SI{16}{\milli\meter} 	& m=&\SI{0.14\pm0.01}{\hertz\per RPM}\\
%								&												& b=&\SI{-160\pm40}{\hertz}\\
%	\alpha = \SI{60}{\degree} 	& \\
%								& \text{Rohrdurchmesser}=\SI{7}{\milli\meter} 	& m=&\SI{0.50\pm0.01}{\hertz\per RPM}\\
%								&												& b=&\SI{-650\pm50}{\hertz}\\
%								& \text{Rohrdurchmesser}=\SI{10}{\milli\meter} 	& m=&\SI{0.22\pm0.08}{\hertz\per RPM}\\
%								&												& b=&\SI{-100\pm340}{\hertz}\\
%								& \text{Rohrdurchmesser}=\SI{16}{\milli\meter} 	& m=&\SI{0.12\pm0.01}{\hertz\per RPM}\\
%								&												& b=&\SI{-116\pm33}{\hertz}\\
%\end{align*}


\subsection{Bestimmung des Strömungsprofils}
Der Winkel des Prismas ist im Folgenden immer $\alpha=\SI{15}{\degree}$.
Die Messwerte für die Messung bei einer Motoreinstellung von $\SI{65}{\percent}$ sind in Tabelle \ref{ström1} aufgetragen.
\begin{table}
    \centering
    \caption{Messwerte für eine Motorleistung von $\SI{65}{\percent}$.}
    \label{ström1}
    \begin{tabular}{S[table-format=2.1] S[table-format=3.0] S[table-format=2.1]}
        \toprule
        {Eindringtiefe$/\si{\micro\second}$} & {$\symup\Delta\nu/\si{\hertz}$} & {$\symup\sigma/\si{\percent}$} \\
        \midrule
12.0   & 140  & 12.2 \\
12.5 & 159  & 12.5 \\
13.0   & 208  & 8.6 \\
13.5 & 287  & 8.6 \\
14.0   & 354  & 4.9 \\
14.5 & 403  & 6.1 \\
15.0   & 452  & 5.2 \\
15.5 & 464  & 4.7 \\
16.0   & 452  & 5.0 \\
16.5 & 427  & 4.1 \\
17.0   & 378  & 6.1 \\
17.5 & 342  & 9.6 \\
18.0   & 342  & 10.3 \\
18.5 & 366  & 7.4 \\
19.0   & 366  & 8.1 \\
19.5 & 354  & 7.8 \\
        \bottomrule
    \end{tabular}
\end{table}
\noindent
Die Messwerte werden entsprechend der verschiedenen Schallgeschwindigkeiten in Prisma, $c_\text{P}=\frac{\SI{10}{\milli\meter}}{\SI{4}{\micro\second}}$\cite{us3},
und Flüssigkeit, $c_\text{Fl}=\frac{\SI{6}{\milli\meter}}{\SI{4}{\micro\second}}$\cite{us3}, in Längen umgerechnet,
wobei das Prisma eine effektive Weglänge von $\SI{30.7}{\milli\meter}$\cite{us3} besitzt.
Die berechneten Werte sind im Diagramm \ref{ström1fig} aufgetragen.
\begin{figure}[H]
  \centering
  \includegraphics{build/strom1.pdf}
  \caption{Strömungsprofil für eine Motorleistung von $\SI{65}{\percent}$.}
  \label{ström1fig}
\end{figure}
\noindent
Die Messwerte der zweiten Messung, mit einer Motorleistung von $\SI{35}{\percent}$, sind in Tabelle \ref{ström2} aufgetragen.
\begin{table}
    \centering
    \caption{Messwerte für eine Motorleistung von $\SI{35}{\percent}$.}
    \label{ström2}
    \begin{tabular}{S[table-format=2.1] S[table-format=3.0] S[table-format=2.1]}
        \toprule
        {Eindringtiefe$/\si{\micro\second}$} & {$\symup\Delta\nu/\si{\hertz}$} & {$\symup{\Delta\Delta}\nu/\si{\percent}$} \\
        \midrule
12.0   & 134  & 8.5 \\
12.5 & 98   & 11.3 \\
13.0   & 98   & 10.1 \\
13.5 & 110  & 9.1 \\
14.0   & 122  & 7.5 \\
14.5 & 146  & 9.5 \\
15.0   & 146  & 6.9 \\
15.5 & 146  & 7.4 \\
16.0   & 146  & 6.3 \\
16.5 & 122  & 8.0 \\
17.0   & 110  & 14.6 \\
17.5 & 122  & 10.1 \\
18.0   & 122  & 8.3 \\
18.5 & 122  & 10.6 \\
19.0   & 122  & 6.9 \\
19.5 & 122  & 6.8 \\
        \bottomrule
    \end{tabular}
\end{table}
\noindent
Die Messwerte werden wie zuvor umgerechnet und sind im Diagramm \ref{ström2fig} aufgetragen.
\begin{figure}[H]
  \centering
  \includegraphics{build/strom2.pdf}
  \caption{Strömungsprofil für eine Motorleistung von $\SI{35}{\percent}$.}
  \label{ström2fig}
\end{figure}
