\section{Auswertung}
\label{sec:Auswertung}
\subsection{Bestimmung der Strömungsgeschwindigkeit}

Der Innendurchmesser des Rohrs bertägt $d = 10\si{\milli\meter}$.
Die für $\Delta\nu$ gemessenen Werte sind in Tabelle\ref{tab} eingetragen.
Dabei werden die Frequenzverschiebungen als Beträge angegeben.
\begin{table}[H]
    \caption{Frequenzänderungen bei Verschiedenen Prismenwinkeln.}
    \label{tab:freq1}
    \centering
    \begin{tabular}{S[table-format=2] S[table-format=2.2(0)e0] S[table-format=2.2(0)e0]  }
        \toprule
        {$\theta/\si{\degree}$} & {$\Delta\nu/\si{\hertz}$} &{$v/\si{\meter\per\second}$} \\
        \midrule
             15 & 110 & 0.29\\
             15 & 171 & 0.46\\
             15 & 208 & 0.54\\
             15 & 220 & 0.57\\
             15 & 220 & 0.57\\
             30 & 220 & 0.29\\
             30 & 287 & 0.39\\
             30 & 305 & 0.41\\
             30 & 385 & 0.52\\
             30 & 385 & 0.52\\
             60 & 305 & 0.24\\
             60 & 433 & 0.34\\
             60 & 531 & 0.41\\
             60 & 433 & 0.34\\
             60 & 647 & 0.50\\
        \bottomrule
    \end{tabular}
\end{table}
\noindent

Die Geschwindigkeit für die Entsprechenden Frequenzverschiebungen werden mithilfe von Gleichung\eqref{eq:geschw}, gemäß

\begin{equation}
  v= \frac{c\cdot\Delta\nu}{2\cdot\nu_0\cos{\alpha}}
\end{equation}
berechnet.
Die Dopplerwinkel werden wie folgt berechnet:
\begin{equation}
  \alpha = 90\si{\degree} - \arcsin(\sin(\theta )\cdot \frac{c_L}{c_P})
\end{equation}
Die erhaltenen Werte sind in Tabelle eingetragen.
\begin{table}[H]
    \caption{Dopplerwinkel.}
    \label{tab:dopp}
    \centering
    \begin{tabular}{S[table-format=2] S[table-format=2.2(0)e0]  }
        \toprule
        {$\theta/\si{\degree}$} & {$\alpha/\si{\degree}$}  \\
        \midrule
             15 & 80.06 \\
             30 & 70.53\\
             60 & 54.74\\

        \bottomrule
    \end{tabular}
\end{table}
\noindent

In Abbildung sind die Geschwindigkeiten gegen den Quotienten aus gemssener Frequenzänderung und dem Kosinus des jeweiligen Dopplerwinkels aufgetragen.
Zudem wird mit SciPy eine lineare Regression mit
\begin{equation}
  f(x)= mx + b
\end{equation}
durgeführt.
\begin{figure}[H]
  \centering
  \includegraphics{build/15.pdf}
  \caption{Frequenzverschiebung in Abhängigkeit der Geschwindigkeit.}
  \label{fig:1mm}
\end{figure}
Die so erhaltenen Parameter sind in Tabelle\ref{tab:par1} aufgetragen.
\begin{table}[H]
    \caption{Parameter der regressions Rechnung.}
    \label{tab:par1}
    \centering
    \begin{tabular}{S[table-format=2] S[table-format=4.8(8)e0] S[table-format=1.2(2)e0] }
        \toprule
        {$\theta/\si{\degree}$} & {$m/\si{\per\meter}$} & {$b\cdot10^-6/\si{\per\second}$}  \\
        \midrule
             15 &  2222.2222146\pm0.0000011 & 12.70\pm1.80 \\
             30 &  2222.2222269\pm0.0000006 & -4.4\pm0.40 \\
             60 &  2222.2222289\pm0.0000012 & -6.3\pm0.70 \\

        \bottomrule
    \end{tabular}
\end{table}
\noindent

\subsection{Bestimmung des Strömungsprofils}
