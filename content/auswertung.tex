\section{Auswertung}
\label{sec:Auswertung}
\subsection{Bestimmung der Strömungsgeschwindigkeit}


Die Dopplerwinkel werden wie folgt berechnet:
\begin{equation}
  \alpha = 90\si{\degree} - \arcsin\left(\sin(\theta )\cdot \frac{c_L}{c_P}\right)
\end{equation}
Die erhaltenen Werte sind in Tabelle \ref{tab:dopp} eingetragen.
\begin{table}[H]
    \caption{Dopplerwinkel.}
    \label{tab:dopp}
    \centering
    \begin{tabular}{S[table-format=2] S[table-format=2.2(0)e0]  }
        \toprule
        {$\theta/\si{\degree}$} & {$\alpha/\si{\degree}$}  \\
        \midrule
             15 & 80.06 \\
             30 & 70.53\\
             60 & 54.74\\

        \bottomrule
    \end{tabular}
\end{table}
\noindent

Die erhaltenen Werte der Messung sind in Tabellen \ref{tab:15}, \ref{tab:30} und \ref{tab:60} eingetragen.


%tabelle


\begin{table}
    \centering
    \caption{Messwerte für einen Winkel von $\alpha = \SI{15}{\degree}$.}
    \label{tab:15}
    \begin{tabular}{S[table-format=4.0] S[table-format=3.0]}
        \toprule
        {Strömungsgeschwindigkeit/RPM} & {$\symup\Delta\nu/\si{\hertz}$} \\
        \midrule
        3120  & 110\\
        3660  & 171\\
        4200  & 208\\
        4750  & 220\\
        5290  & 220\\
        3110  & 256\\
        3650  & 293\\
        4180  & 391\\
        4750  & 439\\
        5300  & 525\\
        3110  & 73\\
        3660  & 85\\
        4200  & 98\\
        4760  & 110\\
        5300  & 122\\

        \bottomrule
    \end{tabular}
\end{table}

\begin{table}
    \centering
    \caption{Messwerte für einen Winkel von $\alpha = \SI{30}{\degree}$.}
    \label{tab:30}
    \begin{tabular}{S[table-format=4.0] S[table-format=3.0]}
        \toprule
        {Strömungsgeschwindigkeit/RPM} & {$\symup\Delta\nu/\si{hertz}$} \\
        \midrule
3120  & 220\\
3660  & 287\\
4200  & 305\\
4750  & 385\\
5290  & 385\\
3110  & 415\\
3650  & 458\\
4180  & 623\\
4750  & 580\\
5300  & 830\\
3110  & 92\\
3660  & 122\\
4200  & 134\\
4760  & 171\\
5300  & 195\\
        \bottomrule
    \end{tabular}
\end{table}

\begin{table}
    \centering
    \caption{Messwerte für einen Winkel von $\alpha = \SI{60}{\degree}$.}
    \label{tab:60}
    \begin{tabular}{S[table-format=4.0] S[table-format=3.0]}
        \toprule
        {Strömungsgeschwindigkeit/RPM} & {$\symup\Delta\nu/\si{hertz}$} \\
        \midrule
3120  & 305\\
3660  & 433\\
4200  & 531\\
4750  & 433\\
5290  & 647\\
3110  & 525\\
3650  & 702\\
4180  & 830\\
4750  & 1001\\
5300  & 1172\\
3110  & 134\\
3660  & 183\\
4200  & 220\\
4760  & 256\\
5300  & 281\\
        \bottomrule
    \end{tabular}
\end{table}

\begin{figure}[H]
  \centering
  \includegraphics{build/15.pdf}
  \caption{Lineare Ausgleichsgerade durch die Messwerte zum Winkel $\alpha = \SI{15}{\degree}$.}
  \label{fig:15}
\end{figure}

\begin{figure}[H]
  \centering
  \includegraphics{build/30.pdf}
  \caption{Lineare Ausgleichsgerade durch die Messwerte zum Winkel $\alpha = \SI{30}{\degree}$.}
  \label{fig:30}
\end{figure}

\begin{figure}[H]
  \centering
  \includegraphics{build/60.pdf}
  \caption{Lineare Ausgleichsgerade durch die Messwerte zum Winkel $\alpha = \SI{60}{\degree}$.}
  \label{fig:60}
\end{figure}



In Abbildung \ref{fig:15}, \ref{fig:30} und \ref{fig:60} sind die Geschwindigkeiten gegen den Quotienten aus gemessener Frequenzänderung und dem Cosinus des jeweiligen
Dopplerwinkels aufgetragen.
Zudem wird mit SciPy eine lineare Regression mit
\begin{equation}
  f(x)= mx + b
\end{equation}
durgeführt.
Es ergeben sich folgende Fit-Parameter:
\begin{align*}
	\alpha = \SI{15}{\degree} 	& \\
								& \text{Rohrdurchmesser}=\SI{7}{\milli\meter} 	& m=&\SI{0.72\pm0.05}{\hertz\per RPM}\\
								&												& b=&\SI{-830\pm230}{\hertz}\\
								& \text{Rohrdurchmesser}=\SI{10}{\milli\meter} 	& m=&\SI{0.29\pm0.08}{\hertz\per RPM}\\
								&												& b=&\SI{-130\pm330}{\hertz}\\
								& \text{Rohrdurchmesser}=\SI{16}{\milli\meter} 	& m=&\SI{0.130\pm0.001}{\hertz\per RPM}\\
								&												& b=&\SI{19\pm5}{\hertz}\\
	\alpha = \SI{30}{\degree} 	& \\
								& \text{Rohrdurchmesser}=\SI{7}{\milli\meter} 	& m=&\SI{0.52\pm0.13}{\hertz\per RPM}\\
								&												& b=&\SI{-400\pm500}{\hertz}\\
								& \text{Rohrdurchmesser}=\SI{10}{\milli\meter} 	& m=&\SI{0.24\pm0.04}{\hertz\per RPM}\\
								&												& b=&\SI{-50\pm160}{\hertz}\\
								& \text{Rohrdurchmesser}=\SI{16}{\milli\meter} 	& m=&\SI{0.14\pm0.01}{\hertz\per RPM}\\
								&												& b=&\SI{-160\pm40}{\hertz}\\
	\alpha = \SI{60}{\degree} 	& \\
								& \text{Rohrdurchmesser}=\SI{7}{\milli\meter} 	& m=&\SI{0.50\pm0.01}{\hertz\per RPM}\\
								&												& b=&\SI{-650\pm50}{\hertz}\\
								& \text{Rohrdurchmesser}=\SI{10}{\milli\meter} 	& m=&\SI{0.22\pm0.08}{\hertz\per RPM}\\
								&												& b=&\SI{-100\pm340}{\hertz}\\
								& \text{Rohrdurchmesser}=\SI{16}{\milli\meter} 	& m=&\SI{0.12\pm0.01}{\hertz\per RPM}\\
								&												& b=&\SI{-116\pm33}{\hertz}\\
\end{align*}


\subsection{Bestimmung des Strömungsprofils}
Der Winkel des Prismas ist im Folgenden immer $\alpha=\SI{15}{\degree}$.
Die Messwerte für die Messung bei einer Motoreinstellung von $\SI{65}{\percent}$ sind in Tabelle \ref{ström1} aufgetragen.
\begin{table}
    \centering
    \caption{Messwerte für eine Motorleistung von $\SI{65}{\percent}$.}
    \label{ström1}
    \begin{tabular}{S[table-format=2.1] S[table-format=3.0] S[table-format=2.1]}
        \toprule
        {Eindringtiefe$/\si{\micro\second}$} & {$\symup\Delta\nu/\si{\hertz}$} & {$\symup{\Delta\Delta}\nu/\si{\percent}$} \\
        \midrule
4    & 342  & 8.5 \\
4.5  & 366  & 7.3 \\
5    & 354  & 6.2 \\
5.5  & 366  & 6.5 \\
6    & 378  & 7.4 \\
6.5  & 366  & 9.1 \\
7    & 360  & 7.2 \\
7.5  & 354  & 8.6 \\
8    & 360  & 6.4 \\
8.5  & 366  & 7.7 \\
9    & 342  & 7.5 \\
9.5  & 354  & 6.7 \\
10   & 342  & 6.1 \\
10.5 & 342  & 7.1 \\
11   & 317  & 7.2 \\
11.5 & 323  & 14.2 \\
12   & 140  & 12.2 \\
12.5 & 159  & 12.5 \\
13   & 208  & 8.6 \\
13.5 & 287  & 8.6 \\
14   & 354  & 4.9 \\
14.5 & 403  & 6.1 \\
15   & 452  & 5.2 \\
15.5 & 464  & 4.7 \\
16   & 452  & 5.0 \\
16.5 & 427  & 4.1 \\
17   & 378  & 6.1 \\
17.5 & 342  & 9.6 \\
18   & 342  & 10.3 \\
18.5 & 366  & 7.4 \\
19   & 366  & 8.1 \\
19.5 & 354  & 7.8 \\
        \bottomrule
    \end{tabular}
\end{table}
\noindent
Die Messwerte werden entsprechend der verschiedenen Schallgeschwindigkeiten in Prisma, $c_\text{P}=\frac{\SI{10}{\milli\meter}}{\SI{4}{\micro\second}}$\cite{us3}, 
und Flüssigkeit, $c_\text{Fl}=\frac{\SI{6}{\milli\meter}}{\SI{4}{\micro\second}}$\cite{us3}, in Längen umgerechnet, 
wobei das Prisma eine effektive Weglänge von $\SI{30.7}{\milli\meter}$\cite{us3} besitzt.
Die berechneten Werte sind im Diagramm \ref{ström1fig} aufgetragen.
\begin{figure}[H]
  \centering
  \includegraphics{build/ström1.pdf}
  \caption{Strömungsprofil für eine Motorleistung von $\SI{65}{\percent}$.}
  \label{ström1fig}
\end{figure}
\noindent
Die Messwerte der zweiten Messung, mit einer Motorleistung von $\SI{35}{\percent}$, sind in Tabelle \ref{ström2} aufgetragen.
\begin{table}
    \centering
    \caption{Messwerte für eine Motorleistung von $\SI{35}{\percent}$.}
    \label{ström2}
    \begin{tabular}{S[table-format=2.1] S[table-format=3.0] S[table-format=2.1]}
        \toprule
        {Eindringtiefe$/\si{\micro\second}$} & {$\symup\Delta\nu/\si{\hertz}$} & {$\symup{\Delta\Delta}\nu/\si{\percent}$} \\
        \midrule
4    & 146  & 9.1 \\
4.5  & 146  & 10.6 \\
5    & 146  & 11.2 \\
5.5  & 159  & 10.3 \\
6    & 159  & 9.3 \\
6.5  & 159  & 7.4 \\
7    & 159  & 9.7 \\
7.5  & 159  & 10.6 \\
8    & 159  & 11.2 \\
8.5  & 159  & 7.3 \\
9    & 153  & 10.2 \\
9.5  & 159  & 8.6 \\
10   & 159  & 9.1 \\
10.5 & 159  & 9.5 \\
11   & 146  & 10.1 \\
11.5 & 146  & 8.6 \\
12   & 134  & 8.5 \\
12.5 & 98   & 11.3 \\
13   & 98   & 10.1 \\
13.5 & 110  & 9.1 \\
14   & 122  & 7.5 \\
14.5 & 146  & 9.5 \\
15   & 146  & 6.9 \\
15.5 & 146  & 7.4 \\
16   & 146  & 6.3 \\
16.5 & 122  & 8.0 \\
17   & 110  & 14.6 \\
17.5 & 122  & 10.1 \\
18   & 122  & 8.3 \\
18.5 & 122  & 10.6 \\
19   & 122  & 6.9 \\
19.5 & 122  & 6.8 \\
        \bottomrule
    \end{tabular}
\end{table}
\noindent
Die Messwerte werden wie zuvor umgerechnet und sind im Diagramm \ref{ström2fig} aufgetragen.
\begin{figure}[H]
  \centering
  \includegraphics{build/ström2.pdf}
  \caption{Strömungsprofil für eine Motorleistung von $\SI{35}{\percent}$.}
  \label{ström2fig}
\end{figure}
