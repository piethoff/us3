\section{Auswertung}
\label{sec:Auswertung}
\subsection{Bestimmung der Strömungsgeschwindigkeit}


Die Dopplerwinkel werden wie folgt berechnet:
\begin{equation}
  \alpha = 90\si{\degree} - \arcsin\left(\sin(\theta )\cdot \frac{c_L}{c_P}\right)
\end{equation}
Die erhaltenen Werte sind in Tabelle \ref{tab:dopp} eingetragen.
\begin{table}[H]
    \caption{Dopplerwinkel.}
    \label{tab:dopp}
    \centering
    \begin{tabular}{S[table-format=2] S[table-format=2.2(0)e0]  }
        \toprule
        {$\theta/\si{\degree}$} & {$\alpha/\si{\degree}$}  \\
        \midrule
             15 & 80.06 \\
             30 & 70.53\\
             60 & 54.74\\

        \bottomrule
    \end{tabular}
\end{table}
\noindent

Die erhaltenen Werte der Messung sind in Tabellen \ref{tab:15}, \ref{tab:30} und \ref{tab:60} eingetragen.


%tabelle


\begin{table}
    \centering
    \caption{Messwerte für einen Winkel von $\alpha = \SI{15}{\degree}$.}
    \label{tab:15}
    \begin{tabular}{S[table-format=4.0] S[table-format=3.0]}
        \toprule
        {Strömungsgeschwindigkeit/RPM} & {$\symup\Delta\nu$} \\
        \midrule

        \bottomrule
    \end{tabular}
\end{table}

\begin{table}
    \centering
    \caption{Messwerte für einen Winkel von $\alpha = \SI{30}{\degree}$.}
    \label{tab:30}
    \begin{tabular}{S[table-format=4.0] S[table-format=3.0]}
        \toprule
        {Strömungsgeschwindigkeit/RPM} & {$\symup\Delta\nu$} \\
        \midrule
3120  & 220\\
3660  & 287\\
4200  & 305\\
4750  & 385\\
5290  & 385\\
3110  & 415\\
3650  & 458\\
4180  & 623\\
4750  & 580\\
5300  & 830\\
3110  & 92\\
3660  & 122\\
4200  & 134\\
4760  & 171\\
5300  & 195\\
        \bottomrule
    \end{tabular}
\end{table}

\begin{table}
    \centering
    \caption{Messwerte für einen Winkel von $\alpha = \SI{60}{\degree}$.}
    \label{tab:60}
    \begin{tabular}{S[table-format=4.0] S[table-format=3.0]}
        \toprule
        {Strömungsgeschwindigkeit/RPM} & {$\symup\Delta\nu$} \\
        \midrule
3120  & 305\\
3660  & 433\\
4200  & 531\\
4750  & 433\\
5290  & 647\\
3110  & 525\\
3650  & 702\\
4180  & 830\\
4750  & 1001\\
5300  & 1172\\
3110  & 134\\
3660  & 183\\
4200  & 220\\
4760  & 256\\
5300  & 281\\
        \bottomrule
    \end{tabular}
\end{table}

\begin{figure}[H]
  \centering
  \includegraphics{build/15.pdf}
  \caption{Lineare Ausgleichsgerade durch die Messwerte zum Winkel $\alpha = \SI{15}{\degree}$.}
  \label{fig:15}
\end{figure}

\begin{figure}[H]
  \centering
  \includegraphics{build/30.pdf}
  \caption{Lineare Ausgleichsgerade durch die Messwerte zum Winkel $\alpha = \SI{30}{\degree}$.}
  \label{fig:30}
\end{figure}

\begin{figure}[H]
  \centering
  \includegraphics{build/60.pdf}
  \caption{Lineare Ausgleichsgerade durch die Messwerte zum Winkel $\alpha = \SI{60}{\degree}$.}
  \label{fig:60}
\end{figure}



In Abbildung \ref{fig:15}, \ref{fig:30} und \ref{fig:60} sind die Geschwindigkeiten gegen den Quotienten aus gemessener Frequenzänderung und dem Cosinus des jeweiligen
Dopplerwinkels aufgetragen.
Zudem wird mit SciPy eine lineare Regression mit
\begin{equation}
  f(x)= mx + b
\end{equation}
durgeführt.
Es ergeben sich folgende Fit-Parameter:
\begin{align*}
	\alpha = \SI{15}{\degree} 	& \\
								& \text{Rohrdurchmesser}=\SI{7}{\milli\meter} 	& m=&\SI{999}{\hertz\per RPM}\\
								&												& b=&\SI{999}{\hertz}\\
								& \text{Rohrdurchmesser}=\SI{10}{\milli\meter} 	& m=&\SI{0.72\pm0.05}{\hertz\per RPM}\\
								&												& b=&\SI{-830\pm230}{\hertz}\\
								& \text{Rohrdurchmesser}=\SI{16}{\milli\meter} 	& m=&\SI{0.130\pm0.001}{\hertz\per RPM}\\
								&												& b=&\SI{19\pm5}{\hertz}\\
	\alpha = \SI{30}{\degree} 	& \\
								& \text{Rohrdurchmesser}=\SI{7}{\milli\meter} 	& m=&\SI{0.24\pm0.04}{\hertz\per RPM}\\
								&												& b=&\SI{-50\pm160}{\hertz}\\
								& \text{Rohrdurchmesser}=\SI{10}{\milli\meter} 	& m=&\SI{0.52\pm0.13}{\hertz\per RPM}\\
								&												& b=&\SI{-400\pm500}{\hertz}\\
								& \text{Rohrdurchmesser}=\SI{16}{\milli\meter} 	& m=&\SI{0.14\pm0.01}{\hertz\per RPM}\\
								&												& b=&\SI{-160\pm40}{\hertz}\\
	\alpha = \SI{60}{\degree} 	& \\
								& \text{Rohrdurchmesser}=\SI{7}{\milli\meter} 	& m=&\SI{0.22\pm0.08}{\hertz\per RPM}\\
								&												& b=&\SI{-100\pm340}{\hertz}\\
								& \text{Rohrdurchmesser}=\SI{10}{\milli\meter} 	& m=&\SI{0.50\pm0.01}{\hertz\per RPM}\\
								&												& b=&\SI{-650\pm50}{\hertz}\\
								& \text{Rohrdurchmesser}=\SI{16}{\milli\meter} 	& m=&\SI{0.12\pm0.01}{\hertz\per RPM}\\
								&												& b=&\SI{-116\pm33}{\hertz}\\
\end{align*}




\subsection{Bestimmung des Strömungsprofils}
