\section{Auswertung}
\label{sec:Auswertung}
\subsection{Bestimmung der Strömungsgeschwindigkeit}


Die Dopplerwinkel werden wie folgt berechnet:
\begin{equation}
  \alpha = 90\si{\degree} - \arcsin\left(\sin(\theta )\cdot \frac{c_L}{c_P}\right)
\end{equation}
Hier ist $c_\text{Fl} = \SI{1800}{\meter\per\second}$\cite{us3} und $c_\text{P} = \SI{2700}{\meter\per\second}$\cite{us3}.
Die erhaltenen Werte sind in Tabelle \ref{tab:dopp} eingetragen.
\begin{table}[H]
    \caption{Dopplerwinkel.}
    \label{tab:dopp}
    \centering
    \begin{tabular}{S[table-format=2] S[table-format=2.2(0)e0]  }
        \toprule
        {$\theta/\si{\degree}$} & {$\alpha/\si{\degree}$}  \\
        \midrule
             15 & 80.06 \\
             30 & 70.53\\
             60 & 54.74\\
        \bottomrule
    \end{tabular}
\end{table}
\noindent Die erhaltenen Werte der Messung sind in Tabellen \ref{tab:15}, \ref{tab:30} und \ref{tab:60} eingetragen.
Die Strömungsgeschwindigkeit wird nach Gleichung \eqref{eq:geschw} mit
\begin{equation}
 v = \frac{c_\text{Fl}  \symup\Delta\nu}{2\nu_0 \cos \alpha}
\end{equation}
berechnet, wobei gilt $\nu_0 = \SI{2}{\mega\hertz}$ und $c_\text{Fl} = \SI{1800}{\meter\per\second}$\cite{us3}.
%tabelle
\begin{table}
    \centering
    \caption{Messwerte für einen Winkel von $\alpha = \SI{15}{\degree}$.}
    \label{tab:15}
    \begin{tabular}{S[table-format=4.0] S[table-format=3.0] S[table-format=1.2] S[table-format=2.0]}
        \toprule
        {Strömungsgeschwindigkeit/RPM} & {$\symup\Delta\nu/\si{\hertz}$} & {$v/\si[per-mode=reciprocal]{\meter\per\second}$} & {Innendurchmesser$/\si{\milli\meter}$} \\
        \midrule
        3110  & 256 & 0.67 & 7\\
        3650  & 293 & 0.76 & 7\\
        4180  & 391 & 1.02 & 7\\
        4750  & 439 & 1.14 & 7\\
        5300  & 525 & 1.37 & 7\\
        3120  & 110 & 0.28 & 10\\
        3660  & 171 & 0.45 & 10\\
        4200  & 208 & 0.54 & 10\\
        4750  & 220 & 0.57 & 10\\
        5290  & 220 & 0.57 & 10\\
        3110  & 73 & 0.19 & 16\\
        3660  & 85 & 0.22 & 16\\
        4200  & 98 & 0.26 & 16\\
        4760  & 110 & 0.29 & 16\\
        5300  & 122 & 0.32 & 16\\

        \bottomrule
    \end{tabular}
\end{table}

\begin{table}
    \centering
    \caption{Messwerte für einen Winkel von $\alpha = \SI{30}{\degree}$.}
    \label{tab:30}
    \begin{tabular}{S[table-format=4.0] S[table-format=3.0] S[table-format=1.2] S[table-format=2.0]}
        \toprule
        {Strömungsgeschwindigkeit/RPM} & {$\symup\Delta\nu/\si{\hertz}$}  & {$v/\si[per-mode=reciprocal]{\meter\per\second}$} & {Innendurchmesser$/\si{\milli\meter}$}\\
        \midrule
3110  & 415 & 0.56 & 7\\
3650  & 458 & 0.62 & 7\\
4180  & 623 & 0.84 & 7\\
4750  & 580 & 0.78 & 7\\
5300  & 830 & 1.12 & 7\\
3120  & 220 & 0.30 & 10\\
3660  & 287 & 0.39 & 10\\
4200  & 305 & 0.41 & 10\\
4750  & 385 & 0.52 & 10\\
5290  & 385 & 0.52 & 10\\
3110  & 92 & 0.12 & 16\\
3660  & 122 & 0.16 & 16\\
4200  & 134 & 0.18 & 16\\
4760  & 171 & 0.23 & 16\\
5300  & 195 & 0.26 & 16\\
        \bottomrule
    \end{tabular}
\end{table}

\begin{table}
    \centering
    \caption{Messwerte für einen Winkel von $\alpha = \SI{60}{\degree}$.}
    \label{tab:60}
    \begin{tabular}{S[table-format=4.0] S[table-format=4.0] S[table-format=1.2] S[table-format=2.0]}
        \toprule
        {Strömungsgeschwindigkeit/RPM} & {$\symup\Delta\nu/\si{\hertz}$} & {$v/\si[per-mode=reciprocal]{\meter\per\second}$} & {Innendurchmesser$/\si{\milli\meter}$} \\
        \midrule
3110  & 525 & 0.41 & 7\\
3650  & 702 & 0.55 & 7\\
4180  & 830 & 0.65 & 7\\
4750  & 1001 & 0.78 & 7\\
5300  & 1172 & 0.91 & 7\\
3120  & 305 & 0.23 & 10\\
3660  & 433 & 0.34 & 10\\
4200  & 531 & 0.41 & 10\\
4750  & 433 & 0.34 & 10\\
5290  & 647 & 0.50 & 10\\
3110  & 134 & 0.10 & 16\\
3660  & 183 & 0.14 & 16\\
4200  & 220 & 0.17 & 16\\
4760  & 256 & 0.20 & 16\\
5300  & 281 & 0.22 & 16\\
        \bottomrule
    \end{tabular}
\end{table}

\begin{figure}[H]
  \centering
  \includegraphics{build/30.pdf}
  \caption{Geschwindigkeit aufgetragen gegen die Frequenzverschiebung geteilt durch die entsprechenden Dopplerwinkel für einen Rohrdurchmesser von $d= 7\si{\milli\meter}$.}
  \label{fig:30}
\end{figure}

\begin{figure}[H]
  \centering
  \includegraphics{build/15.pdf}
  \caption{Geschwindigkeit aufgetragen gegen die Frequenzverschiebung geteilt durch die entsprechenden Dopplerwinkel für einen Rohrdurchmesser von $d= 10\si{\milli\meter}$.}
  \label{fig:15}
\end{figure}

\begin{figure}[H]
  \centering
  \includegraphics{build/60.pdf}
  \caption{Geschwindigkeit aufgetragen gegen die Frequenzverschiebung geteilt durch die entsprechenden Dopplerwinkel für einen Rohrdurchmesser von $d= 16\si{\milli\meter}$.}
  \label{fig:60}
\end{figure}



\noindent In Abbildung \ref{fig:30}, \ref{fig:15} und \ref{fig:60} sind die Geschwindigkeiten gegen den Quotienten aus gemessener Frequenzänderung und dem Cosinus des jeweiligen
Dopplerwinkels aufgetragen.
%Zudem wird mit SciPy eine lineare Regression mit
%\begin{equation}
%  f(x)= mx + b
%\end{equation}
%durgeführt.
%Es ergeben sich folgende Fit-Parameter:
%\begin{align*}
%	\alpha = \SI{15}{\degree} 	& \\
%								& \text{Rohrdurchmesser}=\SI{7}{\milli\meter} 	& m=&\SI{0.72\pm0.05}{\hertz\per RPM}\\
%								&												& b=&\SI{-830\pm230}{\hertz}\\
%								& \text{Rohrdurchmesser}=\SI{10}{\milli\meter} 	& m=&\SI{0.29\pm0.08}{\hertz\per RPM}\\
%								&												& b=&\SI{-130\pm330}{\hertz}\\
%								& \text{Rohrdurchmesser}=\SI{16}{\milli\meter} 	& m=&\SI{0.130\pm0.001}{\hertz\per RPM}\\
%								&												& b=&\SI{19\pm5}{\hertz}\\
%	\alpha = \SI{30}{\degree} 	& \\
%								& \text{Rohrdurchmesser}=\SI{7}{\milli\meter} 	& m=&\SI{0.52\pm0.13}{\hertz\per RPM}\\
%								&												& b=&\SI{-400\pm500}{\hertz}\\
%								& \text{Rohrdurchmesser}=\SI{10}{\milli\meter} 	& m=&\SI{0.24\pm0.04}{\hertz\per RPM}\\
%								&												& b=&\SI{-50\pm160}{\hertz}\\
%								& \text{Rohrdurchmesser}=\SI{16}{\milli\meter} 	& m=&\SI{0.14\pm0.01}{\hertz\per RPM}\\
%								&												& b=&\SI{-160\pm40}{\hertz}\\
%	\alpha = \SI{60}{\degree} 	& \\
%								& \text{Rohrdurchmesser}=\SI{7}{\milli\meter} 	& m=&\SI{0.50\pm0.01}{\hertz\per RPM}\\
%								&												& b=&\SI{-650\pm50}{\hertz}\\
%								& \text{Rohrdurchmesser}=\SI{10}{\milli\meter} 	& m=&\SI{0.22\pm0.08}{\hertz\per RPM}\\
%								&												& b=&\SI{-100\pm340}{\hertz}\\
%								& \text{Rohrdurchmesser}=\SI{16}{\milli\meter} 	& m=&\SI{0.12\pm0.01}{\hertz\per RPM}\\
%								&												& b=&\SI{-116\pm33}{\hertz}\\
%\end{align*}


\subsection{Bestimmung des Strömungsprofils}
Der Winkel des Prismas ist im Folgenden immer $\alpha=\SI{15}{\degree}$.
Die Messwerte für die Messung bei einer Motoreinstellung von $\SI{65}{\percent}$ sind in Tabelle \ref{ström1} aufgetragen, wobei $\sigma$ die Steuintensität von $\symup\Delta\nu$ darstellt.
\begin{table}[H]
    \centering
    \caption{Messwerte für eine Motorleistung von $\SI{65}{\percent}$.}
    \label{ström1}
    \begin{tabular}{S[table-format=2.1] S[table-format=2.2] S[table-format=3.0] S[table-format=1.2] S[table-format=2.1]}
        \toprule
        {Eindringtiefe$/\si{\micro\second}$} &{Eindringtiefe$/\si{\milli\meter}$}& {$\symup\Delta\nu/\si{\hertz}$} & {$v/\si[per-mode=reciprocal]{\meter\per\second}$} & {$\symup\sigma/\si{\percent}$} \\
        \midrule
12.0 & 30.28 & 140  & 0.37 & 12.2 \\
12.5 & 31.03  & 159  & 0.42 & 12.5 \\
13.0 & 31.78  & 208  & 0.55 & 8.6 \\
13.5 & 32.53  & 287  & 0.76 & 8.6 \\
14.0 & 33.28  & 354  & 0.94 & 4.9 \\
14.5 & 34.03  & 403  & 1.07 & 6.1 \\
15.0 & 34.78 & 452  & 1.20 & 5.2\\
15.5 & 35.53  & 464  & 1.23 & 4.7 \\
16.0 & 36.28 & 452  & 1.20 & 5.0 \\
16.5 & 37.03  & 427  & 1.13 &  4.1 \\
17.0 & 37.78 & 378  & 1.00 & 6.1 \\
17.5 & 38.53  & 342  & 0.91 & 9.6 \\
18.0 & 39.28  & 342  & 0.91 & 10.3 \\
18.5 & 40.03  & 366  & 0.97 & 7.4 \\
19.0 & 40.78  & 366  & 0.97 & 8.1 \\
19.5 & 41.53  & 354  & 0.94 & 7.8 \\
        \bottomrule
    \end{tabular}
\end{table}
\noindent
Die Messwerte werden entsprechend der verschiedenen Schallgeschwindigkeiten in Prisma, $c_\text{P}=\frac{\SI{10}{\milli\meter}}{\SI{4}{\micro\second}}$\cite{us3},
und Flüssigkeit, $c_\text{Fl}=\frac{\SI{6}{\milli\meter}}{\SI{4}{\micro\second}}$\cite{us3}, in Längen umgerechnet,
wobei das Prisma eine effektive Weglänge von $\SI{30.7}{\milli\meter}$\cite{us3} besitzt.
Die tatsächliche Weglänge lässt sich somit bestimmen durch:
\begin{equation}
	\label{weglänge}
	x =
	\begin{cases}
		t\cdot c_\text{P}, & t \le \frac{\SI{30.7}{\milli\meter}\text{\cite{us3}}}{c_\text{P}} \\
		t\cdot c_\text{Fl} + \SI{30.7}{\milli\meter}\text{\cite{us3}}\left(1-\frac{c_\text{Fl}}{c_\text{P}}\right), & t > \frac{\SI{30.7}{\milli\meter}\text{\cite{us3}}}{c_\text{P}}
	\end{cases}
\end{equation}
Die berechneten Werte sind im Abbildung \ref{ström1fig} und Abbildung \ref{streu1fig} aufgetragen.
Dabei ist zu beachten, dass Messwerte für eine Endringtiefe über $40\si{\milli\meter}$ für die Auswertung vernachlässigt werden.
Dies wird gemacht, weil das Strömungsrohr einen Durchmesser von $d=10\si{\milli\meter}$ besitzt.
\begin{figure}[H]
  \centering
  \includegraphics{build/strom1.pdf}
  \caption{Strömungsprofil für eine Motorleistung von $\SI{65}{\percent}$.}
  \label{ström1fig}
\end{figure}
\begin{figure}[H]
  \centering
  \includegraphics{build/streu1.pdf}
  \caption{Strömungsprofil für eine Motorleistung von $\SI{65}{\percent}$.}
  \label{streu1fig}
\end{figure}

\noindent Bei beiden Größen ist ein parabelförmiger Verlauf zu erkennen.
Die Parabel der Abweichung ist nach oben und die Parabel der Geschwindigkeit nach unten geöffnet.
Die Geschwindigkeit ist in der Mitte des Rohres am größten und nimmt zu den Seiten ab.
Die Abweichung nimmt bei großen Geschwindigkeiten ab.
Die Messwerte der zweiten Messung, mit einer Motorleistung von $\SI{35}{\percent}$, sind in Tabelle \ref{ström2} aufgetragen, wobei $\sigma$ die Steuintensität von $\symup\Delta\nu$ darstellt.
\begin{table}[H]
    \centering
    \caption{Messwerte für eine Motorleistung von $\SI{35}{\percent}$.}
    \label{ström2}
    \begin{tabular}{S[table-format=2.1] S[table-format=2.2] S[table-format=3.0] S[table-format=1.2] S[table-format=2.1]}
        \toprule
        {Eindringtiefe$/\si{\micro\second}$} &{Eindringtiefe$/\si{\milli\meter}$}& {$\symup\Delta\nu/\si{\hertz}$} & {$v/\si[per-mode=reciprocal]{\meter\per\second}$} & {$\sigma/\si{\percent}$} \\
        \midrule
12.0 & 30.28 & 134  & 0.35 & 8.5 \\
12.5 & 31.03  & 98   & 0.26 &  11.3 \\
13.0 & 31.78  & 98   & 0.26 & 10.1 \\
13.5 & 32.53  & 110  & 0.29 & 9.1 \\
14.0 & 33.28  & 122  & 0.32 & 7.5 \\
14.5 & 34.03  & 146  & 0.39 & 9.5 \\
15.0 & 34.78  & 146  & 0.39 & 6.9 \\
15.5 & 35.53  & 146  & 0.39 & 7.4 \\
16.0 & 36.28  & 146  & 0.39 & 6.3 \\
16.5 & 37.03  & 122  & 0.32 & 8.0 \\
17.0 & 37.78  & 110  & 0.29 & 14.6 \\
17.5 & 38.53  & 122  & 0.32 & 10.1 \\
18.0 & 39.28  & 122  & 0.32 & 8.3 \\
18.5 & 40.03  & 122  & 0.32 & 10.6 \\
19.0 & 40.78  & 122  & 0.32 & 6.9 \\
19.5 & 41.53  & 122  & 0.32 & 6.8 \\
        \bottomrule
    \end{tabular}
\end{table}
\noindent
Die Messwerte werden wie zuvor umgerechnet und sind im Abbildung \ref{ström2fig} und Abbidung \ref{streu2fig} aufgetragen.
\begin{figure}[H]
  \centering
  \includegraphics{build/strom2.pdf}
  \caption{Strömungsprofil für eine Motorleistung von $\SI{35}{\percent}$.}
  \label{ström2fig}
\end{figure}
\begin{figure}[H]
  \centering
  \includegraphics{build/streu2.pdf}
  \caption{Strömungsprofil für eine Motorleistung von $\SI{35}{\percent}$.}
  \label{streu2fig}
\end{figure}
\noindent Ein ähnlicher Verlauf zu Abbildung \ref{ström1fig} und Abbildung \ref{streu1fig} ist zu beobachten.
Der Parabelverlauf ist in Abbildung \ref{ström2fig} und Abbildung \ref{streu2fig} nicht so deutlich zu erkennen.
