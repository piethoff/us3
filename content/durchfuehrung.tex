\section{Durchführung}
\label{sec:Durchführung}
Der Versuchsaufbau besteht aus einem Ultraschall-Doppler-Generator, einer Sonde und einem Rechner für die Datenanalyse.
Die zu untersuchenden Objekte sind Strömungsrohre mit den Innendurchmessern $\SI{7}{\milli\meter}$, $\SI{10}{\milli\meter}$ und $\SI{16}{\milli\meter}$.
Als Flüssigkeit wird ein Gemisch aus Wasser, Glycerin und Glaskugeln verwendet.
Die Flussgeschwindigkeit kann mit einer Zentrifugalpumpe reguliert werden.
Auf den Rohren sind Doppler-Prismen angebracht.
Die Doppler-Prismen besitzen, jeweils drei Einstellflächen.
Die Einstellflächen sind so beschaffen, dass der Einfallswinkel des Ultraschalls jeweils auf $\SI{15}{\degree}$, $\SI{30}{\degree}$ und $\SI{60}{\degree}$ eingestellt werden kann.
\subsection{Bestimmung der Strömungsgeschwindigkeit als Funktion des Dopplerwinkels}
Das $SAMPLE-VOLUME$ des Ultraschall-Generators wird auf $LARGE$ gestellt.
Die $\SI{2}{\mega\hertz}$-Sonde wird mit einem Doppler-Prisma an eines der Strömungsrohre gekoppelt.
Dazu wird ein Koppelmittel verwendet.
Es wird eine Flussgeschwindigkeit an der Pumpe eingestellt.
Die Frequenzverschiebung wird für die möglichen Dopplerwinkel gemessen.
Die Messung wird für vier weitere Geschwindigkeiten wiederholt.
Analog gestaltet sich die Messung für die restlichen Strömungsrohre.
\subsection{Bestimmung des Strömungsprofils}
Das $SAMPLE-VOLUME$ des Ultraschall-Generators wird auf $SMALL$ gestellt.
Die $\SI{2}{\mega\hertz}$-Sonde wird mit dem Doppler-Prisma an die Strömnugsröhre mit dem Innendurchmesser von $d= \SI{10}{\milli\meter}$ gekoppelt.
Der Einfallswinkel wird dabei auf $\SI{15}{\degree}$ gestellt.
Die Strömungsgeschwindigkeit wird auf $\SI{70}{\percent}$ der maximalen Pumpleistung gestellt.
Die Messtiefe wird mit dem Tiefenregler des Ultraschall-Generators auf einen Wert eingestellt, der $\SI{30}{\milli\meter}$ entspricht.
Es werden die Strömungsgeschwindigkeit und die Streuintensität gemessen.
Die Messung wird in $\SI{0.75}{\milli\meter}$-Schritten wiederholt bis die Eindringunstiefe bei $\SI{11}{\milli\meter}$ angelangt ist.
Die Eindringtiefe wird hierfür jeweils über die Schallgeschwindigkeiten für das Prisma,
$c=\SI{2700}{\meter\per\second}$ \cite{prisma}, für die Flüssigkeit im Rohr,
$c=\SI{1800}{\meter\per\second}$ \cite{schlonze}, und der effektiven Weglänge des Prismas, $d=\SI{30.7}{\milli\meter}$, 
in $\si{\micro\second}$ umgewandelt.
