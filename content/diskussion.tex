\section{Diskussion}
\label{sec:Diskussion}
Die Steigungen der Ausgleichsgeraden für die entsprechenden Innendurchmesser der Strömungsrohre liegen sehr nach aneinander.
Auch die Fehler der Steigungen sind gering.
Diese sind ein bis zwei Größenordnunger kleiner als die Steigungen.
\\Der Einfallswinkel hat nur geringe Auswirkungen auf die Messwerte.
Die Abweichungen der Koordinantenabschnitte sind deutlich größer und mindern die Aussagekraft der Messwerte.
Die Messung der Frequenzverschiebung ist problematisch, da die Werte stark schwanken und somit von dem Zeitpunkt des Aufschreibens abhängen.
\\Bei der Untersuchung des Strömungsprofils fällt auf, dass sich für eine hohe Motorleistung ein klares Strömungsprofil abzeichnet, wie es zu erwarten ist.
Am Rand des Rohres ist die Geschwindigkeit gering und steigt zu Mitte hin an, um danach wieder abzufallen.
Für eine kleine Motorleistung ist ein ähnlicher Verlauf nur zu erahnen und die Messwerte entsprechen im Allgemeinen nicht den Erwartungen.
